% -*- root: main.tex -*-

%-------------------------------------------------------------------------------
\chapterimage{chapter_head_2.pdf} 

%-------------------------------------------------------------------------------
\chapter{ROS 명령어}

%-------------------------------------------------------------------------------
\section{ROS 명령어 정리}\index{ROS 명령어 정리}

ROS는 쉘(shell)환경에서 명령어로 파일 시스템 이용, 소스코드 편집, 빌드, 디버깅, 패키지 관리 등을 처리할 수 있다. 로스를 제대로 사용하기 위해서는 기본적인 리눅스 명령어 이외에도 로스만의 고유한 명령어들을 익힐 필요가 있다. 로스 명령어로는 다음과 같이 다양하게 존재한다. 

처음부터 모든 명령어를 숙련하게 사용할 수 없겠지만, 다음의 내용을 보고 필요할 때마다 이용하면 되겠다. 처음에는 익숙하지 않아서 불편하게 느껴질 수 있겠지만, 사용하면 사용할 수록 매우 빠르고 편하게 각종 기능을 명령어로 처리하는 자신의 모습을 보게 될 것이다.

로스의 다양한 명령어를 숙달하기 위하여 본 강좌에서는 우선 각 명령어의 기능을 간단히 설명하고, 하나 하나 예제를 들어가며 소개하도록 하겠다. 그리고 각 명령어의 사용 빈도 및 중요도를 고려하여 필자가 별표로 점수를 매겨두었다. 별표가 많은 명령어의 경우, 사용 빈도수가 높고 중요한 명령어이니 꼭 익혀두기로 하자.

\vspace{\baselineskip}
\noindent
\textbf{[ROS 쉘 명령어]}
\begin{description}
\item[roscd] (★★★) ros+cd(changes directory) : ROS 패키지 또는 스택의 디렉토리 변경 명령어
\item[rospd] (☆☆☆) ros+pushd : ROS 디렉토리 인덱스에 디렉토리 추가
\item[rosd] (☆☆☆) ros+directory : ROS 디렉토리 인덱스 확인 명령어
\item[rosls] (★☆☆) ros+ls(lists files) : ROS 패키지의 파일 리스트를 확인하는 명령어
\item[rosed] (★☆☆) ros+ed(editor) : ROS 패키지의 파일을 편집하는 명령어
\item[roscp] (☆☆☆) ros+cp(copies files) : ROS 패키지의 파일 복사하는 명령어
\end{description}

\vspace{\baselineskip}
\noindent
\textbf{[ROS 실행 명령어]}
\begin{description}
\item[roscore] (★★★) ros+core : master (ROS 네임 서비스) + rosout (stdout/stderr) + parameter server (매개변수관리)
\item[rosrun] (★★★) ros+run : 패키지의 노드를 실행하는 명령어
\item[roslaunch] (★★★) ros+launch : 패키지의 노드를 복수개 실행하는 명령어
\item[rosclean] (★☆☆) ros+clean : ros 로그 파일을 체크하거나 삭제하는 명령어
\end{description}

\vspace{\baselineskip}
\noindent
\textbf{[ROS 정보 명령어]}
\begin{description}
\item[rostopic] (★★★) ros+topic : ROS 토픽 정보를 확인하는 명령어
\item[rosservice] (★★★) ros+service : ROS 서비스 정보를 확인하는 명령어
\item[rosnode] (★★★) ros+node : ROS의 노드 정보를 얻는 명령어
\item[rosparam] (★★★) ros+param(parameter) : ROS 파라미터 정보를 확인, 수정 가능한 명령어
\item[rosmsg] (★★☆) ros+msg : ROS 메세지 선언 정보를 확인하는 명령어
\item[rossrv] (★★☆) ros+srv : ROS 서비스 선언 정보를 확인하는 명령어
\item[roswtf] (☆☆☆) ros+wtf : ROS 시스템을 검사하는 명령어
\item[rosversion] (★☆☆) ros+version : ros 패키및 배포 릴리즈 버전의 정보를 확인하는 명령어
\item[rosbag] (★★★) ros+bag : ROS 메세지를 기록, 재생하는 명령어
\end{description}

\vspace{\baselineskip}
\noindent
\textbf{[ROS 캐킨 명령어]}
\begin{description}
\item[catkin\_create\_pkg] (★★★) 캐킨 빌드 시스템에 의한 패키지 자동 생성
\item[catkin\_eclipse] (★★☆) 캐킨 빌드 시스템에 의해 생성된 패키지를 이클립스에서 사용할 수 있도록 변경하는 명령어
\item[catkin\_find] (☆☆☆) 캐킨 검색
\item[catkin\_generate\_changelog] (☆☆☆) 캐킨 변경로그 생성
\item[catkin\_init\_workspace] (★☆☆) 캐킨 빌드 시스템의 작업폴더 초기화
\item[catkin\_make] (★★★) 캐킨 빌드 시스템을 기반으로한 빌드 명령어
\end{description}

\vspace{\baselineskip}
\noindent
\textbf{[ROS 패키지 명령어]}
\begin{description}
\item[rosmake] (☆☆☆) ros+make : ROS package 를 빌드한다. (구 ROS 빌드 시스템에서 사용됨)
\item[rosinstall] (★☆☆) ros+install : ROS 추가 패키지 설치 명령어
\item[roslocate] (☆☆☆) ros+locate : ROS 패키지 정보 관련 명령어 
\item[roscreate-pkg] (☆☆☆) ros+create-pkg : ROS 패키지를 자동 생성하는 명령어 (구 ROS 빌드 시스템에서 사용됨)
\item[rosdep] (★☆☆) ros+dep(endencies) : 해당 패키지의 의존성 파일들을 설치하는 명령어
\item[rospack] (★★☆) ros+pack(age) : ROS 패키지와 관련된 정보를 알아보는 명령어
\end{description}

%-------------------------------------------------------------------------------
\section{ROS 쉘 명령어}\index{ROS 쉘 명령어}

로스 쉘 명령어는 일명 rosbash라고 부른다. 이는 리눅스에서 일반적으로 사용하는 bash 쉘 명령어를 로스에서 사용하는 방법이다. 주로 ros 라는 접두사에 cd, pd, d, ls, ed, cp, run 등의 접미사를 붙여서 사용하게 된다. 이와 관련된 명령어들을 아래에 소개한다.  

\vspace{\baselineskip}
\noindent
\begin{description}
\item[roscd] (★★★) ros + cd (changes directory) 
\item[rospd] (☆☆☆) ros + pushd 
\item[rosd] (☆☆☆) ros + directory
\item[rosls] (★☆☆) ros + ls (lists files)
\item[rosed] (★☆☆) ros + ed (editor)
\item[roscp] (☆☆☆) ros+cp (copies files)
\item[rosrun] (★★★) ros+run 
\end{description}

\vspace{\baselineskip}
\noindent
이 중, 비교적 자주 사용되는 roscd, rosls, rosed 명령어에 대해서 자세히 알아보자.

\noindent
※ rosrun 은 rosbash 에 포함되나 의미상 로스 실행 명령어이기 때문에 로스 실행 명령어에서 다루도록 한다.

%-------------------------------------------------------------------------------
\subsection{roscd : 로스 디렉토리 이동}\index{roscd : 로스 디렉토리 이동}

① 사용법
\begin{lstlisting}[language=bash]
roscd [%*패키지이름*)]
\end{lstlisting}

\noindent
② 사용예
\begin{lstlisting}[language=bash]
$ roscd turtlesim
\end{lstlisting}

\noindent
③ 결과값
\begin{lstlisting}[language=bash]
/opt/ros/indigo/share/turtlesim$
\end{lstlisting}

\vspace{\baselineskip}
\noindent
②번에서 처럼, turtlesim 이라는 패키지 이름을 매개변수로 넣어주고 실행하게 되면 지정된 패키지가 저장된 폴더로 이동하게 된다. ③의 현재 결과에서는 turtlesim 이라는 패키지가 로스가 설치되어 있는 폴더에 있으므로 위와 같이 나왔으나, 자신이 작성한 패키지의 이름을 매개변수로 넣어주면 자신이 설정한 패키지의 디렉토리로 이동할 수도 있다. 명령어 기반의 로스를 사용하는데 있어서 매우 사용빈도가 높은 명령어이다.

\vspace{\baselineskip}
\noindent
※ 위와 같은 실행 및 결과를 위해서는 관련 패키지인 ros-indigo-turtlesim 패키지가 설치되어 있어야 한다. 만약 설치되어 있지 않은 경우 설치하도록 하자. 설치 명령어는 sudo apt-get install ros-indigo-turtlesim  이다.

%-------------------------------------------------------------------------------
\subsection{rosls : 로스 리스트 파일}\index{rosls : 로스 리스트 파일}

① 사용법
\begin{lstlisting}[language=bash]
rosls [%*패키지이름*)]
\end{lstlisting}

\noindent
② 사용예
\begin{lstlisting}[language=bash]
$ rosls turtlesim
\end{lstlisting}

\noindent
③ 결과값
\begin{lstlisting}[language=bash]
$ rosls turtlesim/
cmake  images  msg  package.xml  srv
\end{lstlisting}

\vspace{\baselineskip}
\noindent
지정한 ROS 패키지의 파일 리스트를 확인하는 명령어이다. roscd 명령어를 이용하여 해당 패키지로 이동후에 일반 ls 명령어로 같은 기능을 수행할 수도 있지만, 간혹 바로 확인할 필요가 있을경우에 사용된다. 사용 빈도수는 매우 떨어진다.

%-------------------------------------------------------------------------------
\subsection{rosed : 로스 편집 명령어}\index{rosed : 로스 편집 명령어}

① 사용법
\begin{lstlisting}[language=bash]
rosed [%*패키지이름*)] [%*파일이름*)]
\end{lstlisting}

\noindent
② 사용예
\begin{lstlisting}[language=bash]
$ rosed turtlesim package.xml 
\end{lstlisting}

\vspace{\baselineskip}
\noindent
turtlesim 패키지의 package.xml 를 편집하고자 할때 사용하는 명령어이다. 이를 실행하면 사용자가 설정한 에디터로 해당 파일을 열게된다. 급하게 간단히 파일을 수정하고자 할때 사용된다. 사용되는 에디터는 "~/.bashrc" 파일에 export EDITOR='emacs -nw' 와 같이 지정하여 사용가능하다. 필자는 비교적 단순한 편집일 경우 이를 사용한다. 프로그램 작성과 같은 작업 rosed 이외에 자신만의 개발환경을 사용하기를 추천한다.

\vspace{\baselineskip}
\noindent
③ 결과값
\begin{lstlisting}[language=bash]
%*사용자가 지정해둔 에디터를 이용하여 해당 파일을 열게된다.*)
\end{lstlisting}

\vspace{\baselineskip}
\noindent
바로 명령어창에서 수정이 필요한 단순한 작업에서 많이 사용되지만, 그 이외의 프로그램 작업등에는 비추천이다. 전체적인 사용 빈도수는 매우 떨어진다.

%-------------------------------------------------------------------------------
\section{ROS 실행 명령어}\index{ROS 실행 명령어}

로스 실행 명령어는 로스 노드의 실행을 주관한다. 무엇보다 필수는 roscore로 노드간의 네임 서버로 사용된다. 그리고 실행명령어로는 rosrun 및 roslaunch가 있다. rosrun 는 하나의 노드를 실행하게 되며, roslaunch 는 하나이상의 노드를 실행할때 사용된다. 그리고 rosclean 는 노드 실행시 기록되는 로그의 삭제에 사용되는 명령어이다.

\vspace{\baselineskip}
\noindent
\begin{description}
\item[roscore] (★★★) ros+core : master (ROS 네임 서비스) + rosout (stdout/stderr) + parameter server (매개변수관리)
\item[rosrun] (★★★) ros+run : 패키지의 노드를 실행하는 명령어
\item[roslaunch] (★★★) ros+launch : 패키지의 노드를 복수개 실행하는 명령어
\item[rosclean] (★☆☆) ros+clean : ros 로그 파일을 체크하거나 삭제하는 명령어
\end{description}

%-------------------------------------------------------------------------------
\subsection{roscore : 로스 코어 실행}\index{roscore : 로스 코어 실행}

로스코어는 노드들간의 메시지 통신에서 연결 정보를 관리하는 마스터로서, 로스를 사용하기 위해서 제일 먼저 구동해야하는 필수 요소이다. 다음과 같이 "roscore"라는 실행 명령어로 로스 마스터는 구동되며, XMLRPC으로 서버를 구동하게 된다. 마스터는 노드간의 접속을 위하여 노드들의 이름, 토픽 및 서비스의 이름, 메시지 형태, URI 주소 및 포트를  등록받고, 요청이 있을 경우 이 정보를 다른 노드에게 알려주는 역할을 한다. 

\vspace{\baselineskip}
\noindent
① 사용법
\begin{lstlisting}[language=bash]
roscore [%*옵션*)]
\end{lstlisting}

\noindent
② 사용예
\begin{lstlisting}[language=bash]
$ roscore
\end{lstlisting}

\noindent
위의 명령어로 로스 코어를 실행하게되면 사용자가 설정해둔 ROS\_MASTER\_URI 를 마스터 URI로 하여, 마스터를 기동하게 된다. ROS\_MASTER\_URI 은 "~/.bashrc" 에서 사용자가 설정하도록 되어있다.

\vspace{\baselineskip}
\noindent
③ 결과값
\begin{lstlisting}[language=bash]
$ roscore
... logging to /home/rts/.ros/log/a42b1130-63dc-11e4-a5a1-74d02bc77892/roslaunch-rts-3547.log
Checking log directory for disk usage. This may take awhile.
Press Ctrl-C to interrupt
Done checking log file disk usage. Usage is <1GB.
started roslaunch server http://192.168.4.1:57385/
ros_comm version 1.11.9

SUMMARY
========
PARAMETERS
 * /rosdistro: indigo
 * /rosversion: 1.11.9

NODES

auto-starting new master
process[master]: started with pid [3560]
ROS_MASTER_URI=http://192.168.4.1:11311/

setting /run_id to a42b1130-63dc-11e4-a5a1-74d02bc77892
process[rosout-1]: started with pid [3575]
started core service [/rosout]
\end{lstlisting}

\vspace{\baselineskip}
\noindent
위 결과에서 /home/rt/.ros/log/ 폴더에 로그들이 저장되고 있다는 것과 "Ctrl-C" 키로 로스 코어를 종료할 수 있다느 것, roslaunch server, ROS\_MASTER\_URI 등의 정보, /rosdistro 및 /rosversion의 파라미터 서버, /rosout 의 서비스, /rosout 노드가 실행되었음을 알수 있다.


%-------------------------------------------------------------------------------
\subsection{rosrun : 로스 노드 실행}\index{rosrun : 로스 노드 실행}

지정한 패키지의 하나의 노드를 실행하는 명령어이다.

\vspace{\baselineskip}
\noindent
① 사용법
\begin{lstlisting}[language=bash]
rosrun [%*패키지이름*)] [%*노드이름*)]
\end{lstlisting}

\noindent
② 사용예
\begin{lstlisting}[language=bash]
$ rosrun turtlesim turtlesim_node 
\end{lstlisting}

\noindent
turtlesim 이라는 패키지의 turtlesim\_node 이라는 노드를 실행하는 명령어이다. rosrun은 지정한 패키지의 하나의 노드를 실행시킨다. 

\vspace{\baselineskip}
\noindent
③ 결과값
\begin{lstlisting}[language=bash]
$ rosrun turtlesim turtlesim_node 
[INFO] [1383445615.677782380]: Starting turtlesim with node name /turtlesim
[INFO] [1383445615.686475328]: Spawning turtle [turtle1] at x=[5.544445], y=[5.544445], theta=[0.000000]
\end{lstlisting}




























%-------------------------------------------------------------------------------