%-------------------------------------------------------------------------------
\chapterimage{chapter_head_2.pdf} % Chapter heading image

%-------------------------------------------------------------------------------
\chapter{로봇 운영 체제}

%-------------------------------------------------------------------------------
\section{로봇 소프트웨어 플랫폼}\index{로봇 소프트웨어 플랫폼}

%-------------------------------------------------------------------------------
\subsection{플랫폼이 가져온 변화}\index{플랫폼이 가져온 변화}

잠깐, 로봇 이야기가 아닌, 핸드폰에 대해서 이야기해보자. 불가 20년전만 하더라도 스마트폰과 같은 핸드폰이 지금과 같이 많은 사람들의 필수품이 되리라고는 생각지도 못했었다.
더욱이, 초기의 핸드폰은 핸드폰이라는 이름이 무상할 정도로 크고 무거운 무전기와 같았고, 가격은 상상을 초월할 정도로 비쌌으며, 성능은 떨어졌다.
이에 비하여 현재의 스마트폰은 가격도 저렴하고, 가볍고, 작고 사용하기도 편리하다는 것을 본다면 그 변화는 놀랍기까지 하다.

올해 2013년, 40주년이 된 세계 최초의 상용 핸드폰인 모토롤라 다이나택 8000의 경우\footnote{http://www.segye.com/content/html/2013/04/04/20130404004829.html}\footnote{CarToyBlog, "Happy 40th Birthday to the Cell Phone!",  http://blog.cartoys.com/date/2013/04/},\footnote{http://venma.tistory.com/entry/Motorola-DynaTAC-8000X}, 그 당시 가격으로 500만원대, 0.8Kg의 무게, 최대 통화시간 30분, 충전 10시간 소요, 최대 30개 전화번호 입력 정도만이 가능하였다.
그 뒤로 핸드폰 시장은 급성장하게 되었고, 현재의 스마트폰까지 이르렀다. 그리고, 지금은 우리에게는 없어서는 안되는 생활 필수 아이템이 되었다.

그렇다면 이러한 것이 가능했던 이유는 무엇이였을까? 

초기에 많은 핸드폰 관련 회사들은 기술 경쟁을 거듭하며, 슬림하면서도 가볍고 고화질, 고음질, 휴대성 등 자신들만의 특징들을 네세우며 새로운 기기들을 내놓았다.
그 당시만해도 핸드폰은 모델마다 하드웨어에 맞게 기능을 추가하기 위하여 하드웨어 의존적인 펌웨어 개발을 했다.
한 회사에만 수 십가지 기종이 있었을테고, 전세계적으로는 어땠을까?
상상할 수도 없는 규모이다.
목적은 다 비슷한데 통합되지 못하고 그렇게 개발자들은 죽어라~ 새로운 하드웨어 개발 일정에 맞추어 펌웨어 개발을 했고 이는 반복되었다.
새로운 하드웨어마다 의존적으로 개발하였기에 그 발전 속도는 떨어졌고, 의존성으로 인하여 관리 비용은 높기만 했다.
그리고 그 비용은 우리 소비자의 몫이였다. 

지금은 어떠한가?
안드로이드, iOS, 윈도우즈 등 대표적인 OS를 기반으로 개발자들이 힘을 모으게 되었고 이러한 소프트웨어 플랫폼을 기반으로 핸드폰이라는 하드웨어 플랫폼을 잘 몰라도 관련 어플리케이션 개발에 문제가 없게 되었다.
그리고, 앱 개발자라는 새로운 소프트웨어 직종을 생길 정도로 스마트폰 운영체제를 기반으로 한 개발 환경이 확립되었으며, 스마트폰 운영체제 관리팀 이외에도 스마트폰 회사, 앱 개발자, 심지어 일반 사용자들까지도 하나로 뭉쳐서 하드웨어와 소프트웨어를 통합하는 플랫폼은 진화의 진화를 거듭하게 되었다.
이는 핸드폰 뿐만아니라 개인컴퓨터 개발과 더불어 불어닥친 유닉스, 리눅스, 윈도우, OS X 등의 컴퓨터용 OS 도 마찬가지 양상이라고 볼 수 있다.

이러한 이유로는 하드웨어의 급성장과 필연적인 사용자들의 수요도 있었겠지만,  소프트웨어 플랫폼 기반으로 지식이 한데 모아져서 나온 결과라고 볼 수 있다.
이러한 소프트웨어 플랫폼은 하드웨어 플랫폼의 인터페이스를 통합시키게 만들고, 나아가 하드웨어를 몰라도 상위 단의 프로그램인 응용 프로그램에 집중할 수 있게 되었기 때문에 사용자들의 수요에 맞는 응용 제품이 나올 수 있었다고 생각된다. 

%-------------------------------------------------------------------------------
\subsection{플랫폼이 가져온 변화}\index{플랫폼이 가져온 변화}

최근, 로봇계도 마찬가지의 움직임을 보이고 있다. 스마트폰 OS 나 개인컴퓨터 OS 에 비하여 그 규모는 작고, 아직 발전 단계이기는 하지만 로봇 소프트웨어 플랫폼은 춘추전국시대라고 볼 수 있을 정도로 매우 활발하게 진행되고 있다.
아래의 리스트는 그 중에서 돋보이는 활동을 보이고 있는 그룹들의 로봇 소프트웨어 플랫폼이다.

\begin{itemize}
\item ERSP,Evolution robotics (회사)\footnote{http://www.evolution.com/products/ersp/}
\item MSRDS,Microsoft (회사)\footnote{http://msdn.microsoft.com/en-us/robotics/default.aspx}
\item OpenRTM,일본의 AIST (국립 연구소)\footnote{http://www.openrtm.org}
\item ROS,Open Source Robotics Foundation\footnote{http://www.osrfoundation.org/}
\item OROCOS,유럽\footnote{http://www.orocos.org/}
\item OPRoS,한국\footnote{http://www.opros.or.kr/}
\end{itemize}

위와 같이 많은 플랫폼들이 등장하고 있고, 현재도 다양한 접근방식으로 증가 추세에 있다.
너도나도 로봇계의 로봇 소프트웨에 플랫폼의 선두에 서고 싶기 때문일 것이다.
현재, 다양한 로봇 소프트웨어 플랫폼이 나오고 있지만, 어느 것이 좋다라고는 섣불리 말하기 어려운게 사실이다.
그러나, 언젠가는 지금의 운영체제들처럼 독특한 스타일은 있을지는 몰라도 사용자 측면에서는 궁극적으로 비슷한 기능들로 압축 될 것이라고 생각한다.
우리는 소프트웨어 플랫폼 자체를 만드는 것이 아닌, 범용적인 로봇 소프트웨어 플랫폼에서도 돌아갈 수 있는 응용 프로그램 개발 능력에 집중하면 좋을 듯 싶다.
예를들어 안드로이드 어플 개발과 iOS 어플 개발이 비슷해진 것을 예를 들 수 있다. 

그렇다면 우리는 현재 나와있는 로봇 소프트웨어 플랫폼 중에서 어떤것을 우선적으로 익혀두면 좋을까?
필자는 이 중 어떤 것이 나에게 제일 적합하고, 앞으로 장래성이 있을까?
그리고 커뮤니케이션은?
오픈소스인가?
개발 환경은 어떻게 지원될까?
이러한 질문에서 가장 정답으로 생각하는 것은 Open Source Robotics Foundation 의 ROS라고 생각한다.
특히, 커뮤니티의 활발성, 준비되어 있는 라이브러리, 확장성, 개발 편의성을 생각해본다면 ROS 만한 것도 없어 보인다.
이 춘추전국시대와 같은 다양한 로봇 소프트웨어 플랫폼 중에서 어떤것이 살아 남을지 필자도 매우 궁금하다.

%-------------------------------------------------------------------------------
\subsection{로봇 소프트웨어 플랫폼이 가져올 미래}\index{로봇 소프트웨어 플랫폼이 가져올 미래}

로봇 소프트웨어 플랫폼은 정해진 하드웨어 플랫폼을 기준으로 하고 있기에 소프트웨어 플랫폼을 이용하면 하드웨어에 대한 지식이 없어도 응용 프로그램을 작성 가능하다.
이는 최신 스마트폰의 하드웨어 구성 및 세부 내역을 몰라도 어플을 작성가능한 것과 마찬가지이다.
또한, 로봇 개발자가 하드웨어 설계부터 소프트웨어 설계까지 하던 이전 작업 프로세서와 구별되어 더 많은 소프트웨어 인력들이 로봇 응용 제품에 참여 할 수 있다.
즉, 소프트웨어 플랫폼의 역할로 많은 이들이 로봇 개발에 동참하게 되는 길을 열고, 하드웨어는 소프트웨어 플랫폼을 사용하기 위하여 소프트웨어 플랫폼에서 제안하는 인터페이스에 맞도록 설계가 될것이다.
이는 로봇 개발이 급속도로 발전 할 수 있는 계기를 마련하게 되는 것이라고 생각한다.




% \begin{enumerate}
% \item The first item
% \item The second item
% \item The third item
% \end{enumerate}

% \subsection{Bullet Points}\index{Lists!Bullet Points}

% \begin{itemize}
% \item The first item
% \item The second item
% \item The third item
% \end{itemize}

% \subsection{Descriptions and Definitions}\index{Lists!Descriptions and Definitions}

% \begin{description}
% \item[Name] Description
% \item[Word] Definition
% \item[Comment] Elaboration
% \end{description}